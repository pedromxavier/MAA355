\documentclass[brazil]{homework}

\title{MAA355 - Lista 1}
\author{Pedro Maciel Xavier}
\register{116023847}

\begin{document}
    \maketitle*%%

    \quest*{%%
    Seja $V$ um $K$-espaço vetorial, $W_1, W_2 \subset V$ subespaços tais que $W_1 + W_2 = V$ e $W_1 \cap W_2 = \set{0}$. Mostre que $\forall \alpha \in V$ $\alpha$ se escreve de forma única $\alpha = \alpha_1 + \alpha_2, \alpha_1 \in W_1, \alpha_2 \in W_2$.%%
    }%%

    \begin{answer}
        Vamos supor por absurdo que $\exists \alpha \in V$ que não se escreve de forma única como $\alpha = \alpha_1 + \alpha_2, \alpha_1 \in W_1, \alpha_2 \in W_2$,
        isto é, $\alpha = \beta_1 + \beta_2, \alpha_1 \neq \beta_1 \in W_1, \alpha_2 \neq \beta_2 \in W_2$.

        Logo, $\beta_1 + \beta_2 = \alpha_1 + \alpha_2$ e, portanto, $\gamma = \alpha_1 - \beta_1 = \alpha_2 - \beta_2$. É certo que, como $W_i$ são espaços vetoriais, $\alpha_i - \beta_i \in W_i$, $i = 1, 2$. Por fim, $\gamma \neq 0$ pertence tanto a $W_1$ como a $W_2$ e $\set{\gamma} \subseteq W_1 \cap W_2$ é a contradição que buscávamos. \qed
    \end{answer}

    \quest*{%%
    Seja $V$ um $K$-espaço vetorial. Mostre que se $\exists \alpha_1, \dots, \alpha_n \in V$ tais que $\left< \alpha_1, \dots, \alpha_n \right> = V$ então $\text{dim}_K\ V < \infty$.%%
    }%%

    \begin{answer}
        Se $V$ for um $K$-espaço vetorial de dimensão infinita, 
    \end{answer}

    \quest*{%%
    Seja $V$ um $K$-espaço vetorial, $\vet{T} \in \text{End}_K\ V$. Prove que a) e b) são equivalentes.%%
    %%
    \begin{enumerate}[label=\alph*)]
        \item $\text{Im}\ \vet{T} \cap \text{ker}\ \vet{T} = \set{0}$
        \item Se $\vet{T}^2 \alpha = 0$ então $\vet{T} \alpha = 0$
    \end{enumerate}
    }%%

    \begin{answer}%%
        \begin{enumerate}[topsep=0cm, leftmargin=\parindent]
            \item [] a) $\implies$ b) $\vet{T}^2 \alpha = 0$ é o mesmo que dizer que $\vet{T} \left( \vet{T} \alpha \right) = 0$, ou seja, $\vet{T} \alpha \in \text{ker}\,\vet{T}$. Naturalmente, $\vet{T} \alpha \in \text{Im}\, \vet{T}$ e, portanto, $\vet{T} \alpha \in \text{Im}\ \vet{T} \cap \text{ker}\ \vet{T} = \set{0}$. Logo, se $\vet{T}^2 \alpha = 0$ então $\vet{T} \alpha = 0$.
            %%
            %%
            \item [] a) $\impliedby$ b) Seja $\beta \in \text{Im}\ \vet{T} \cap \text{ker}\ \vet{T}$. Como $\beta \in \text{Im}\,\vet{T}$, $\beta = \vet{T} \alpha$ para algum $\alpha \in V$. Por outro lado, $\beta \in \text{ker}\,\vet{T}$ significa que $\vet{T} \beta = 0$. Por conseguinte, $\vet{T} \beta = \vet{T} \left( \vet{T} \alpha \right) = \vet{T}^2 \alpha = 0$ então $\beta = \vet{T} \alpha = 0$. Isso conclui a prova. \qed
        \end{enumerate}
    \end{answer}

    \quest*{Seja $V$ um $K$-espaço vetorial com $\text{dim}_K\ V < \infty$ e $\vet{T} \in \text{End}_K\ V$. Suponha que o posto de $\vet{T}^2$ é o mesmo que o posto de $\vet{T}$. Mostre que
    $$\text{Im}\ \vet{T} \cap \text{ker}\ \vet{T} = \set{0}$$%%
    }%%

    \quest*{%%
    Sejam $m, n \ge 1$ inteiros, $K$ um corpo e $f_1, \dots, f_m \in \left(K^n\right)^{\ast}.$\\
    Para $\alpha \in K^n$ definimos%%
    $$\vet{T} \alpha = \left(f_1 \alpha, \dots, f_m \alpha\right)$$%%
    Mostre que $\vet{T} : K^n \to K^m$ é uma transformação $K$-linear. Mostre também que $\forall \vet{T} \in \text{Hom}_K\left(K^n, K^n\right)$ é desta forma.
    }%%
\end{document}